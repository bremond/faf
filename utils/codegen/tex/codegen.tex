\documentclass[a4paper]{article}

\usepackage{listing}
\usepackage{caption}
\usepackage{minted}
\usepackage{RR}
\usepackage{a4wide} 

\RRdate{\today}
\RRauthor{Maurice Br\'emond {\tt
    maurice.bremond@inria.fr}}
\authorhead{}
\RRtitle{A code generation
  procedure with verifications for the computation of generalized jacobians}
\RRetitle{}
\titlehead{}
\RRresume{}
\RRabstract{Given the expression of a nonsmooth
  function in Sympy, a Python computer algebra system, an element of its
  generalized jacobian may be expressed as a piecewise function. This
  piecewise function takes its values on the smooth domain, where it is the
  smooth jacobian computed by the computer algebra system, and on the
  nonsmooth domain where some values must be provided explicitely. With the
  help of a small tool a C code is generated from this piecewise function. If
  the expression of the generalized jacobian is correct, for inputs in a
  bounded domain, the computation of the jacobian must gives finites
  results. This may be proven with the help of added ACSL static assertions to
  give hints to the frama-c value analysis tools and its subsequents proofs
  obligations.}
\RRmotcle{}
\RRkeyword{}
\RRprojet{Bipop}
\URRhoneAlpes
\RCGrenoble

\PassOptionsToPackage{dvipsnames,svgnames}{xcolor}   
\renewcommand{\baselinestretch}{1.5}
\usepackage{natbib}


\usepackage{geometry}
\usepackage{layout}
\geometry{
  left=2.5cm,
  right=2.5cm,
  top=3.5cm,
  bottom=3cm
}

\usepackage{natbib}
% For unicode 
%\usepackage{ucs}
\usepackage[utf8]{inputenc}
%\usepackage[utf8]{inputenc}
\usepackage[T1]{fontenc}
\usepackage[english]{babel}

% invisible unicode space
\DeclareUnicodeCharacter{00A0}{~}

\usepackage{mathtools, amsthm, amssymb}

\usepackage{graphicx}
\usepackage{psfrag}

%\usepackage{subfigure}
\usepackage{subfig,float}
\usepackage{rotating}

% table package
\usepackage{hhline}
\usepackage{multicol} % for column
\usepackage{tabularx} % for beautiful column

% math packages 
\usepackage{amsmath,amsthm,amsfonts,amssymb,amsbsy,stmaryrd} 

\usepackage{subeqnarray}



% pour la cesure des mots
%\usepackage[T1]{fontenc}
%\usepackage{inputenc}
\usepackage{textcomp}

%pour l'environnement remarque
%\usepackage{theorem}
%\usepackage{ntheorem}

%pour faire des symboles d\'ebiles
\usepackage{pifont}
%pour faire des minis tables des matieres

\usepackage{algorithm}
\usepackage{algorithmic}

%\usepackage[numbers]{natbib}

\renewcommand{\cite}[1]{\protect\citep{#1}}
%\def\cite#1{\citep{#1}}
%\usepackage{draftcopy}

\usepackage{float}
%\usepackage{showkeys}
\usepackage{showlabels}

\usepackage{array}



\usepackage{tikz}
\usepackage{xkeyval,tkz-base}
\usetikzlibrary{arrows}
\usetikzlibrary{calc}

\usepackage{fancyhdr}
\usepackage{lastpage}


\usepackage[pdfencoding=unicode,pdftex,backref=page]{hyperref}






../../../TeX/Report/macro.tex

%%
\begin{document}
%%
\RRNo{123456789}
%\makeRR   % cas d'un rapport de recherche
\makeRT
%% a partir d'ici, chacun fait comme il le souhaite

\newpage
\tableofcontents
\newpage

\section{Introduction}
The computation of numerical solutions of miscellaneous nonsmooth physical
problems may need the expression and the computation of Clarke Generalized
gradients \cite{Clarke1975}.  Code writing for such jacobians can be tedious
and error prone. Errors in the computation may not be obvious to detect when
the jacobian is used in nonsmooth Newton method.


\section{The Method}


The method:

1. Express nonsmooth function in a computer algebra system (Maple, Sagemath, Sympy)

2. Compute the jacobian on differentiable points.

3. Express generalized jacobian as a piecewise function : continuous : ``normal jacobian'',  non differentiable point : a limit

4. perform common subexpression elimination up to functions sqrt, Abs, Max, Heaviside, etc.

5. produce c code with static assertions (value analysis fails without them)

6. static verification with frama-c:

output a proof of: limited domain (physically admissible domain) for inputs ==> is\_finite(result)

\section{Examples}

\subsection{Euclidian norm function}

\subsubsection{code generation from the mathematical definition}
\begin{equation}
  \begin{array}{ll}
  \left [\begin{array}{c}
    x\\
    y
    \end{array} \right] \mapsto \sqrt{x^2+y^2}
  \end{array}
\end{equation}

\begin{equation}
  \begin{array}{ll}
    \left [
      \begin{array}{c}
        x\\
        y
      \end{array} \right]
    \mapsto \left\{
        \begin{array}{ll}
          \left[
            \begin{array}{c}
            \frac{x}{\sqrt{x^2+y^2}}\\
            \frac{y}{\sqrt{x^2+y^2}}
          \end{array} \right] & \mbox{if } (x, y) \in \RR^{\star 2} \\
        \{ v \in \RR^2, ||v|| = 1 \} & \mbox{if } (x, y) = (0, 0)
        
      \end{array}
    \right.
  \end{array}
\end{equation}

\begin{listing}[H] 
  \begin{minted}[frame=single]{python}
    from sympy import Symbol, Matrix, sqrt
    x = Symbol('x', real=True)
    y = Symbol('y', real=True)
    norm = Matrix([sqrt(x*x + y*y)])
  \end{minted}
  \caption{The two-norm function over $\RR^2$ with Sympy}
\end{listing}


The smooth jacobian

\begin{listing}[H] 
  \begin{minted}[frame=single]{python}
    v = Matrix([x, y])
    J_ = f.jacobian(v)
  \end{minted}
  \caption{the smooth jacobian}
\end{listing}

In zero
\begin{listing}[H] 
  \begin{minted}[frame=single]{python}
    from sympy import limit
    def lim0(expr):
        return limit(expr.subs(x,t).subs(y,t), t, 0)
  \end{minted}
  \caption{a limit}
\end{listing}

\begin{listing}[H] 
  \begin{minted}[frame=single]{python}
    from sympy import Piecewise
    J = Matrix(J_.shape[0], J_.shape[1], 
           lambda i, j: Piecewise(
               (J_[i, j], v.norm() > 0.),
               (lim0(J_[i, j]), v.norm() <= 0.)))
  \end{minted}
  \caption{a limit}
\end{listing}


\begin{equation}
  \begin{array}{ll}
    \left [
      \begin{array}{c}
        x\\
        y
      \end{array} \right]
    \mapsto \left\{
        \begin{array}{ll}
          \left[
            \begin{array}{c}
            \frac{x}{\sqrt{x^2+y^2}}\\
            \frac{y}{\sqrt{x^2+y^2}}
          \end{array} \right] & \mbox{if } (x, y) \in \RR^{\star 2} \\
        \left[
          \begin{array}{c}
            \frac{\sqrt{2}}{2}\\
            \frac{\sqrt{2}}{2}
          \end{array} \right] & \mbox{if } (x, y) = (0, 0)
        
      \end{array}
    \right.
  \end{array}
\end{equation}


\begin{listing}[H] 
  \begin{minted}[frame=single]{c}
void norm2d_jacobian(
    double x,
    double y,
    double *result)
{
    double x1 = 0.;
    int x4 = 0;
    double x2 = 0.;
    double x3 = 0.;
    x1 = sqrt(x*x + y*y);
    x4 = x1 > 0.;
    int x5 = 0;
    x5 = x1 <= 0.;
    if (x4)
    {
        x2 = 1.0/x1;
        x3 = 1.0*x2;
    }
    if (x4)
    {
        result[0] = x*x3;
    }
    else if (x5)
    {
        result[0] = 0.707106781186547524400844362104849039284835937688474;
    }
    if (x4)
    {
        result[1] = x3*y;
    }
    else if (x5)
    {
        result[1] = 0.707106781186547524400844362104849039284835937688474;
    }
}
\end{minted}
\caption{a limit}
\end{listing}

\subsubsection{this function may overflow}

\subsubsection{a corrected jacobian}

\section{Jacobians}

The Heaviside function is noted $\theta$

The $\epsilon$ parameter is important to avoid infinites or not a number values. 
The verification is performed by the value analysis.

\subsection{The Alart Curnier and the Jean Moreau functions}

Given the following notations:

\begin{equation}
  \begin{array}{c}
    \Delta_n = r_n -\rho_n  u_n\\
    \Delta_{t1} = t_{t1} - \rho_{t1}  u_{t1}\\
    \Delta_{t2} = t_{t2} - \rho_{t2}  u_{t2}\\
    \Delta_{t} = \left [
      \begin{array}{c}
        \Delta_{t1}\\
        \Delta_{t2}
      \end{array} \right]
  \end{array}
\end{equation}

\subsubsection{Alart Curnier}

The Alart Curnier function $F_{AC}$ is written as:

\begin{equation}
  F_{AC} = \left[\begin{matrix}r_{n} - \max\left(\Delta_{n}, \epsilon\right)\\\begin{cases} \rho_{{t1}} u_{{t1}} & \text{for}\: \lVert {\Delta_{t} \rVert} \leq \mu \max\left(\Delta_{n}, \epsilon\right) \\- \frac{\Delta_{{t1}} \mu \max\left(\Delta_{n}, \epsilon\right)}{\lVert {\Delta_{t} \rVert}} + r_{{t1}} & \text{for}\: \lVert {\Delta_{t} \rVert} > \mu \max\left(\Delta_{n}, \epsilon\right) \end{cases}\\\begin{cases} \rho_{{t2}} u_{{t2}} & \text{for}\: \lVert {\Delta_{t} \rVert} \leq \mu \max\left(\Delta_{n}, \epsilon\right) \\- \frac{\Delta_{{t2}} \mu \max\left(\Delta_{n}, \epsilon\right)}{\lVert {\Delta_{t} \rVert}} + r_{{t2}} & \text{for}\: \lVert {\Delta_{t} \rVert} > \mu \max\left(\Delta_{n}, \epsilon\right) \end{cases}\end{matrix}\right]
\end{equation}

We have 
\begin{equation}
  A = \left[ A_0 A_1 A_2 \right]
\end{equation}

with:
\begin{equation}
  A_0 = \left[\begin{matrix}\rho_{n} \theta\left(- \epsilon - \rho_{n} u_{n} + r_{n}\right)\\\begin{cases} 0 & \text{for}\: \lVert {\Delta_{t} \rVert} \leq \mu \max\left(\Delta_{n}, \epsilon\right) \\\frac{\Delta_{{t1}} \rho_{n} \mu}{\lVert {\Delta_{t} \rVert}} \theta\left(- \epsilon - \rho_{n} u_{n} + r_{n}\right) & \text{for}\: \lVert {\Delta_{t} \rVert} > \mu \max\left(\Delta_{n}, \epsilon\right) \end{cases}\\\begin{cases} 0 & \text{for}\: \lVert {\Delta_{t} \rVert} \leq \mu \max\left(\Delta_{n}, \epsilon\right) \\\frac{\Delta_{{t2}} \rho_{n} \mu}{\lVert {\Delta_{t} \rVert}} \theta\left(- \epsilon - \rho_{n} u_{n} + r_{n}\right) & \text{for}\: \lVert {\Delta_{t} \rVert} > \mu \max\left(\Delta_{n}, \epsilon\right) \end{cases}\end{matrix}\right]
\end{equation}

\begin{equation}
  A_1 = \left[\begin{matrix}0\\\begin{cases} \rho_{{t1}} & \text{for}\: \lVert {\Delta_{t} \rVert} \leq \mu \max\left(\Delta_{n}, \epsilon\right) \\\frac{\rho_{{t1}} \mu \max\left(\Delta_{n}, \epsilon\right)}{\lVert {\Delta_{t} \rVert}^{3}} \left(- \Delta_{{t1}}^{2} + \lVert {\Delta_{t} \rVert}^{2}\right) & \text{for}\: \lVert {\Delta_{t} \rVert} > \mu \max\left(\Delta_{n}, \epsilon\right) \end{cases}\\\begin{cases} 0 & \text{for}\: \lVert {\Delta_{t} \rVert} \leq \mu \max\left(\Delta_{n}, \epsilon\right) \\- \frac{\Delta_{{t1}} \Delta_{{t2}} \rho_{{t1}} \mu \max\left(\Delta_{n}, \epsilon\right)}{\lVert {\Delta_{t} \rVert}^{3}} & \text{for}\: \lVert {\Delta_{t} \rVert} > \mu \max\left(\Delta_{n}, \epsilon\right) \end{cases}\end{matrix}\right]
\end{equation}

\begin{equation}
  A_2 = \left[\begin{matrix}0\\\begin{cases} 0 & \text{for}\: \lVert {\Delta_{t} \rVert} \leq \mu \max\left(\Delta_{n}, \epsilon\right) \\- \frac{\Delta_{{t1}} \Delta_{{t2}} \rho_{{t2}} \mu \max\left(\Delta_{n}, \epsilon\right)}{\lVert {\Delta_{t} \rVert}^{3}} & \text{for}\: \lVert {\Delta_{t} \rVert} > \mu \max\left(\Delta_{n}, \epsilon\right) \end{cases}\\\begin{cases} \rho_{{t2}} & \text{for}\: \lVert {\Delta_{t} \rVert} \leq \mu \max\left(\Delta_{n}, \epsilon\right) \\\frac{\rho_{{t2}} \mu \max\left(\Delta_{n}, \epsilon\right)}{\lVert {\Delta_{t} \rVert}^{3}} \left(- \Delta_{{t2}}^{2} + \lVert {\Delta_{t} \rVert}^{2}\right) & \text{for}\: \lVert {\Delta_{t} \rVert} > \mu \max\left(\Delta_{n}, \epsilon\right) \end{cases}\end{matrix}\right]
\end{equation}

and

\begin{equation}
  B = \left[ B_0 B_1 B_2 \right]
\end{equation}

with:
\begin{equation}
  B_0 = \left[\begin{matrix}- \theta\left(- \epsilon - \rho_{n} u_{n} + r_{n}\right) + 1\\\begin{cases} 0 & \text{for}\: \lVert {\Delta_{t} \rVert} \leq \mu \max\left(\Delta_{n}, \epsilon\right) \\- \frac{\Delta_{{t1}} \mu}{\lVert {\Delta_{t} \rVert}} \theta\left(- \epsilon - \rho_{n} u_{n} + r_{n}\right) & \text{for}\: \lVert {\Delta_{t} \rVert} > \mu \max\left(\Delta_{n}, \epsilon\right) \end{cases}\\\begin{cases} 0 & \text{for}\: \lVert {\Delta_{t} \rVert} \leq \mu \max\left(\Delta_{n}, \epsilon\right) \\- \frac{\Delta_{{t2}} \mu}{\lVert {\Delta_{t} \rVert}} \theta\left(- \epsilon - \rho_{n} u_{n} + r_{n}\right) & \text{for}\: \lVert {\Delta_{t} \rVert} > \mu \max\left(\Delta_{n}, \epsilon\right) \end{cases}\end{matrix}\right]
\end{equation}

\begin{equation}
  B_1 = \left[\begin{matrix}0\\\begin{cases} 0 & \text{for}\: \lVert {\Delta_{t} \rVert} \leq \mu \max\left(\Delta_{n}, \epsilon\right) \\\frac{\Delta_{{t1}}^{2} \mu \max\left(\Delta_{n}, \epsilon\right)}{\lVert {\Delta_{t} \rVert}^{3}} + 1 - \frac{\mu \max\left(\Delta_{n}, \epsilon\right)}{\lVert {\Delta_{t} \rVert}} & \text{for}\: \lVert {\Delta_{t} \rVert} > \mu \max\left(\Delta_{n}, \epsilon\right) \end{cases}\\\begin{cases} 0 & \text{for}\: \lVert {\Delta_{t} \rVert} \leq \mu \max\left(\Delta_{n}, \epsilon\right) \\\frac{\Delta_{{t1}} \Delta_{{t2}} \mu \max\left(\Delta_{n}, \epsilon\right)}{\lVert {\Delta_{t} \rVert}^{3}} & \text{for}\: \lVert {\Delta_{t} \rVert} > \mu \max\left(\Delta_{n}, \epsilon\right) \end{cases}\end{matrix}\right]
\end{equation}

\begin{equation}
  B_2 = \left[\begin{matrix}0\\\begin{cases} 0 & \text{for}\: \lVert {\Delta_{t} \rVert} \leq \mu \max\left(\Delta_{n}, \epsilon\right) \\\frac{\Delta_{{t1}} \Delta_{{t2}} \mu \max\left(\Delta_{n}, \epsilon\right)}{\lVert {\Delta_{t} \rVert}^{3}} & \text{for}\: \lVert {\Delta_{t} \rVert} > \mu \max\left(\Delta_{n}, \epsilon\right) \end{cases}\\\begin{cases} 0 & \text{for}\: \lVert {\Delta_{t} \rVert} \leq \mu \max\left(\Delta_{n}, \epsilon\right) \\\frac{\Delta_{{t2}}^{2} \mu \max\left(\Delta_{n}, \epsilon\right)}{\lVert {\Delta_{t} \rVert}^{3}} + 1 - \frac{\mu \max\left(\Delta_{n}, \epsilon\right)}{\lVert {\Delta_{t} \rVert}} & \text{for}\: \lVert {\Delta_{t} \rVert} > \mu \max\left(\Delta_{n}, \epsilon\right) \end{cases}\end{matrix}\right]
\end{equation}

\subsubsection{Jean Moreau}
The Jean Moreau function $F_{JM}$ is written as:

\begin{equation}
  F_{JM} = \left[\begin{matrix}r_{n} - \max\left(\Delta_{n}, \epsilon\right)\\\begin{cases} \rho_{{t1}} u_{{t1}} & \text{for}\: \lVert {\Delta_{t} \rVert} \leq \mu \max\left(\epsilon, r_{n}\right) \\- \frac{\Delta_{{t1}} \mu \max\left(\epsilon, r_{n}\right)}{\lVert {\Delta_{t} \rVert}} + r_{{t1}} & \text{for}\: \lVert {\Delta_{t} \rVert} > \mu \max\left(\epsilon, r_{n}\right) \end{cases}\\\begin{cases} \rho_{{t2}} u_{{t2}} & \text{for}\: \lVert {\Delta_{t} \rVert} \leq \mu \max\left(\epsilon, r_{n}\right) \\- \frac{\Delta_{{t2}} \mu \max\left(\epsilon, r_{n}\right)}{\lVert {\Delta_{t} \rVert}} + r_{{t2}} & \text{for}\: \lVert {\Delta_{t} \rVert} > \mu \max\left(\epsilon, r_{n}\right) \end{cases}\end{matrix}\right]
\end{equation}

We have 
\begin{equation}
  A = \left[ A_0 A_1 A_2 \right]
\end{equation}

with:
\begin{equation}
  A_0 = \left[\begin{matrix}\rho_{n} \theta\left(- \epsilon - \rho_{n} u_{n} + r_{n}\right)\\0\\0\end{matrix}\right]
\end{equation}

\begin{equation}
  A_1 = \left[\begin{matrix}0\\\begin{cases} \rho_{{t1}} & \text{for}\: \lVert {\Delta_{t} \rVert} \leq \mu \max\left(\epsilon, r_{n}\right) \\\frac{\rho_{{t1}} \mu \max\left(\epsilon, r_{n}\right)}{\lVert {\Delta_{t} \rVert}^{3}} \left(- \Delta_{{t1}}^{2} + \lVert {\Delta_{t} \rVert}^{2}\right) & \text{for}\: \lVert {\Delta_{t} \rVert} > \mu \max\left(\epsilon, r_{n}\right) \end{cases}\\\begin{cases} 0 & \text{for}\: \lVert {\Delta_{t} \rVert} \leq \mu \max\left(\epsilon, r_{n}\right) \\- \frac{\Delta_{{t1}} \Delta_{{t2}} \rho_{{t1}} \mu \max\left(\epsilon, r_{n}\right)}{\lVert {\Delta_{t} \rVert}^{3}} & \text{for}\: \lVert {\Delta_{t} \rVert} > \mu \max\left(\epsilon, r_{n}\right) \end{cases}\end{matrix}\right]
\end{equation}

\begin{equation}
  A_2 = \left[\begin{matrix}0\\\begin{cases} 0 & \text{for}\: \lVert {\Delta_{t} \rVert} \leq \mu \max\left(\epsilon, r_{n}\right) \\- \frac{\Delta_{{t1}} \Delta_{{t2}} \rho_{{t2}} \mu \max\left(\epsilon, r_{n}\right)}{\lVert {\Delta_{t} \rVert}^{3}} & \text{for}\: \lVert {\Delta_{t} \rVert} > \mu \max\left(\epsilon, r_{n}\right) \end{cases}\\\begin{cases} \rho_{{t2}} & \text{for}\: \lVert {\Delta_{t} \rVert} \leq \mu \max\left(\epsilon, r_{n}\right) \\\frac{\rho_{{t2}} \mu \max\left(\epsilon, r_{n}\right)}{\lVert {\Delta_{t} \rVert}^{3}} \left(- \Delta_{{t2}}^{2} + \lVert {\Delta_{t} \rVert}^{2}\right) & \text{for}\: \lVert {\Delta_{t} \rVert} > \mu \max\left(\epsilon, r_{n}\right) \end{cases}\end{matrix}\right]
\end{equation}

and

\begin{equation}
  B = \left[ B_0 B_1 B_2 \right]
\end{equation}

with:
\begin{equation}
  B_0 = \left[\begin{matrix}- \theta\left(- \epsilon - \rho_{n} u_{n} + r_{n}\right) + 1\\\begin{cases} 0 & \text{for}\: \lVert {\Delta_{t} \rVert} \leq \mu \max\left(\epsilon, r_{n}\right) \\- \frac{\Delta_{{t1}} \mu \theta\left(- \epsilon + r_{n}\right)}{\lVert {\Delta_{t} \rVert}} & \text{for}\: \lVert {\Delta_{t} \rVert} > \mu \max\left(\epsilon, r_{n}\right) \end{cases}\\\begin{cases} 0 & \text{for}\: \lVert {\Delta_{t} \rVert} \leq \mu \max\left(\epsilon, r_{n}\right) \\- \frac{\Delta_{{t2}} \mu \theta\left(- \epsilon + r_{n}\right)}{\lVert {\Delta_{t} \rVert}} & \text{for}\: \lVert {\Delta_{t} \rVert} > \mu \max\left(\epsilon, r_{n}\right) \end{cases}\end{matrix}\right]
\end{equation}

\begin{equation}
  B_1 = \left[\begin{matrix}0\\\begin{cases} 0 & \text{for}\: \lVert {\Delta_{t} \rVert} \leq \mu \max\left(\epsilon, r_{n}\right) \\\frac{\Delta_{{t1}}^{2} \mu \max\left(\epsilon, r_{n}\right)}{\lVert {\Delta_{t} \rVert}^{3}} + 1 - \frac{\mu \max\left(\epsilon, r_{n}\right)}{\lVert {\Delta_{t} \rVert}} & \text{for}\: \lVert {\Delta_{t} \rVert} > \mu \max\left(\epsilon, r_{n}\right) \end{cases}\\\begin{cases} 0 & \text{for}\: \lVert {\Delta_{t} \rVert} \leq \mu \max\left(\epsilon, r_{n}\right) \\\frac{\Delta_{{t1}} \Delta_{{t2}} \mu \max\left(\epsilon, r_{n}\right)}{\lVert {\Delta_{t} \rVert}^{3}} & \text{for}\: \lVert {\Delta_{t} \rVert} > \mu \max\left(\epsilon, r_{n}\right) \end{cases}\end{matrix}\right]
\end{equation}

\begin{equation}
  B_2 = \left[\begin{matrix}0\\\begin{cases} 0 & \text{for}\: \lVert {\Delta_{t} \rVert} \leq \mu \max\left(\epsilon, r_{n}\right) \\\frac{\Delta_{{t1}} \Delta_{{t2}} \mu \max\left(\epsilon, r_{n}\right)}{\lVert {\Delta_{t} \rVert}^{3}} & \text{for}\: \lVert {\Delta_{t} \rVert} > \mu \max\left(\epsilon, r_{n}\right) \end{cases}\\\begin{cases} 0 & \text{for}\: \lVert {\Delta_{t} \rVert} \leq \mu \max\left(\epsilon, r_{n}\right) \\\frac{\Delta_{{t2}}^{2} \mu \max\left(\epsilon, r_{n}\right)}{\lVert {\Delta_{t} \rVert}^{3}} + 1 - \frac{\mu \max\left(\epsilon, r_{n}\right)}{\lVert {\Delta_{t} \rVert}} & \text{for}\: \lVert {\Delta_{t} \rVert} > \mu \max\left(\epsilon, r_{n}\right) \end{cases}\end{matrix}\right]
\end{equation}

\subsection{The normal map function}

With the notations:
\begin{equation}
  \begin{array}{c}
    \Delta_{t1} = \mu u_{t1}-r_{t1}\\
    \Delta_{t2} = \mu u_{t2}-r_{t2}\\
    \Delta_{t} = \left [
      \begin{array}{c}
        \Delta_{t1}\\
        \Delta_{t2}
      \end{array} \right]
  \end{array}
\end{equation}

\begin{equation}
  F_{nat} = \left[\begin{matrix}\mu r_{n} - \frac{1}{2} \left(\max\left(0, \lambda_{1}\right) + \max\left(0, \lambda_{2}\right)\right)\\\begin{cases} - \frac{\Delta_{1} \left(\max\left(0, \lambda_{1}\right) - \max\left(0, \lambda_{2}\right)\right)}{2 \lVert {\Delta} \rVert} + r_{{t1}} & \text{for}\: \lVert {\Delta} \rVert > \epsilon \\r_{{t1}} & \text{for}\: \lVert {\Delta} \rVert \leq \epsilon \end{cases}\\\begin{cases} - \frac{\Delta_{2} \left(\max\left(0, \lambda_{1}\right) - \max\left(0, \lambda_{2}\right)\right)}{2 \lVert {\Delta} \rVert} + r_{{t2}} & \text{for}\: \lVert {\Delta} \rVert > \epsilon \\r_{{t2}} + \frac{1}{2} \left(\max\left(0, \lambda_{1}\right) - \max\left(0, \lambda_{2}\right)\right) & \text{for}\: \lVert {\Delta} \rVert \leq \epsilon \end{cases}\end{matrix}\right]
\end{equation}

\begin{equation}
  A_0 = \left[\begin{matrix}\frac{1}{2} \left(\theta\left(\lambda_{1}\right) + \theta\left(\lambda_{2}\right)\right)\\\begin{cases} \frac{\Delta_{1} \left(\theta\left(\lambda_{1}\right) - \theta\left(\lambda_{2}\right)\right)}{2 \lVert {\Delta} \rVert} & \text{for}\: \lVert {\Delta} \rVert > \epsilon \\0 & \text{for}\: \lVert {\Delta} \rVert \leq \epsilon \end{cases}\\\begin{cases} \frac{\Delta_{2} \left(\theta\left(\lambda_{1}\right) - \theta\left(\lambda_{2}\right)\right)}{2 \lVert {\Delta} \rVert} & \text{for}\: \lVert {\Delta} \rVert > \epsilon \\- \frac{1}{2} \left(\theta\left(\lambda_{1}\right) - \theta\left(\lambda_{2}\right)\right) & \text{for}\: \lVert {\Delta} \rVert \leq \epsilon \end{cases}\end{matrix}\right]
\end{equation}

\begin{equation}
  A_{01} = \begin{cases} \frac{\sqrt{2} \mu}{2} \theta\left(\lambda^{{\prime\prime}}_{1}\right) & \text{for}\: \lVert {r_{{t}} \rVert} \leq \epsilon \wedge \lVert {u_{{t}} \rVert} \leq \epsilon \\\frac{\mu r_{{t1}} \theta\left(\lambda^{{\prime}}_{1}\right)}{\lVert {r_{{t}} \rVert}} & \text{for}\: \lVert {\Delta} \rVert \leq \epsilon \\- \frac{\theta\left(\lambda_{1}\right)}{2} \left(- \frac{\Delta_{1} \mu}{\lVert {\Delta} \rVert} - \frac{\mu u_{{t1}}}{\lVert {u_{{t}} \rVert}}\right) - \frac{\theta\left(\lambda_{2}\right)}{2} \left(\frac{\Delta_{1} \mu}{\lVert {\Delta} \rVert} - \frac{\mu u_{{t1}}}{\lVert {u_{{t}} \rVert}}\right) & \text{for}\: \lVert {\Delta} \rVert > \epsilon \wedge \lVert {u_{{t}} \rVert} > \epsilon \\\frac{\mu}{4 \lVert {r_{{t}} \rVert}} \left(- 2 r_{{t1}} \theta\left(\lambda^{{\prime}}_{1}\right) + 2 r_{{t1}} \theta\left(\lambda^{{\prime}}_{2}\right) + \sqrt{2 r_{{t1}}^{2} + 2 r_{{t2}}^{2}} \theta\left(\lambda^{{\prime}}_{1}\right) + \sqrt{2 r_{{t1}}^{2} + 2 r_{{t2}}^{2}} \theta\left(\lambda^{{\prime}}_{2}\right)\right) & \text{for}\: \lVert {u_{{t}} \rVert} \leq \epsilon \wedge \lVert {\Delta} \rVert > \epsilon \wedge \lVert {r_{{t}} \rVert} > \epsilon \end{cases}
\end{equation}

\begin{equation}
  A_{02} = \begin{cases} \frac{\sqrt{2} \mu}{2} \theta\left(\lambda^{{\prime\prime}}_{1}\right) & \text{for}\: \lVert {r_{{t}} \rVert} \leq \epsilon \wedge \lVert {u_{{t}} \rVert} \leq \epsilon \\\frac{\mu r_{{t2}} \theta\left(\lambda^{{\prime}}_{1}\right)}{\lVert {r_{{t}} \rVert}} & \text{for}\: \lVert {\Delta} \rVert \leq \epsilon \wedge \lVert {r_{{t}} \rVert} > \epsilon \wedge \lVert {u_{{t}} \rVert} > \epsilon \\- \frac{\theta\left(\lambda_{1}\right)}{2} \left(- \frac{\Delta_{2} \mu}{\lVert {\Delta} \rVert} - \frac{\mu u_{{t2}}}{\lVert {u_{{t}} \rVert}}\right) - \frac{\theta\left(\lambda_{2}\right)}{2} \left(\frac{\Delta_{2} \mu}{\lVert {\Delta} \rVert} - \frac{\mu u_{{t2}}}{\lVert {u_{{t}} \rVert}}\right) & \text{for}\: \lVert {\Delta} \rVert > \epsilon \wedge \lVert {u_{{t}} \rVert} > \epsilon \\\frac{0.25 \mu}{\lVert {r_{{t}} \rVert}} \left(- 2 r_{{t2}} \theta\left(\lambda^{{\prime}}_{1}\right) + 2 r_{{t2}} \theta\left(\lambda^{{\prime}}_{2}\right) + \sqrt{2 r_{{t1}}^{2} + 2 r_{{t2}}^{2}} \theta\left(\lambda^{{\prime}}_{1}\right) + \sqrt{2 r_{{t1}}^{2} + 2 r_{{t2}}^{2}} \theta\left(\lambda^{{\prime}}_{2}\right)\right) & \text{for}\: \lVert {u_{{t}} \rVert} \leq \epsilon \wedge \lVert {\Delta} \rVert > \epsilon \wedge \lVert {r_{{t}} \rVert} > \epsilon \end{cases}
\end{equation}

\begin{equation}
  A_{11} = \begin{cases} \frac{\mu \left(\mu - 1\right)^{2} \theta\left(\lambda^{{\prime\prime}}_{1}\right)}{2 \mu^{2} - 2 \mu + 1} & \text{for}\: \lVert {r_{{t}} \rVert} \leq \epsilon \wedge \lVert {u_{{t}} \rVert} \leq \epsilon \\\frac{\mu r_{{t1}}^{2} \theta\left(\lambda^{{\prime}}_{1}\right)}{\lVert {r_{{t}} \rVert}^{2}} & \text{for}\: \lVert {\Delta} \rVert \leq \epsilon \\- \frac{\Delta_{1} \theta\left(\lambda_{1}\right)}{2 \lVert {\Delta} \rVert} \left(- \frac{\Delta_{1} \mu}{\lVert {\Delta} \rVert} - \frac{\mu u_{{t1}}}{\lVert {u_{{t}} \rVert}}\right) + \frac{\Delta_{1} \theta\left(\lambda_{2}\right)}{2 \lVert {\Delta} \rVert} \left(\frac{\Delta_{1} \mu}{\lVert {\Delta} \rVert} - \frac{\mu u_{{t1}}}{\lVert {u_{{t}} \rVert}}\right) - \frac{\max\left(0, \lambda_{1}\right)}{2} \left(- \frac{\Delta_{1}^{2} \mu}{\lVert {\Delta} \rVert^{3}} + \frac{\mu}{\lVert {\Delta} \rVert}\right) - \frac{\max\left(0, \lambda_{2}\right)}{2} \left(\frac{\Delta_{1}^{2} \mu}{\lVert {\Delta} \rVert^{3}} - \frac{\mu}{\lVert {\Delta} \rVert}\right) & \text{for}\: \lVert {\Delta} \rVert > \epsilon \wedge \lVert {u_{{t}} \rVert} > \epsilon \\\frac{\mu}{4 \lVert {r_{{t}} \rVert}^{3}} \left(2 \lVert {r_{{t}} \rVert} r_{{t1}}^{2} \theta\left(\lambda^{{\prime}}_{1}\right) + 2 \lVert {r_{{t}} \rVert} r_{{t1}}^{2} \theta\left(\lambda^{{\prime}}_{2}\right) - \sqrt{2} r_{{t1}}^{3} \theta\left(\lambda^{{\prime}}_{1}\right) + \sqrt{2} r_{{t1}}^{3} \theta\left(\lambda^{{\prime}}_{2}\right) - \sqrt{2} r_{{t1}} r_{{t2}}^{2} \theta\left(\lambda^{{\prime}}_{1}\right) + \sqrt{2} r_{{t1}} r_{{t2}}^{2} \theta\left(\lambda^{{\prime}}_{2}\right) - 2 r_{{t2}}^{2} \max\left(0, \lambda^{{\prime}}_{1}\right) + 2 r_{{t2}}^{2} \max\left(0, \lambda^{{\prime}}_{2}\right)\right) & \text{for}\: \lVert {u_{{t}} \rVert} \leq \epsilon \wedge \lVert {\Delta} \rVert > \epsilon \wedge \lVert {r_{{t}} \rVert} > \epsilon \end{cases}
\end{equation}

\begin{equation}
  A_{12} = \begin{cases} \frac{\mu^{2} \left(\mu - 1\right) \theta\left(\lambda^{{\prime\prime}}_{1}\right)}{2 \mu^{2} - 2 \mu + 1} & \text{for}\: \lVert {r_{{t}} \rVert} \leq \epsilon \wedge \lVert {u_{{t}} \rVert} \leq \epsilon \\\frac{\mu r_{{t1}} r_{{t2}} \theta\left(\lambda^{{\prime}}_{1}\right)}{\lVert {r_{{t}} \rVert}^{2}} & \text{for}\: \lVert {\Delta} \rVert \leq \epsilon \\\frac{\Delta_{1} \Delta_{2} \mu \max\left(0, \lambda_{1}\right)}{2 \lVert {\Delta} \rVert^{3}} - \frac{\Delta_{1} \Delta_{2} \mu \max\left(0, \lambda_{2}\right)}{2 \lVert {\Delta} \rVert^{3}} - \frac{\Delta_{1} \theta\left(\lambda_{1}\right)}{2 \lVert {\Delta} \rVert} \left(- \frac{\Delta_{2} \mu}{\lVert {\Delta} \rVert} - \frac{\mu u_{{t2}}}{\lVert {u_{{t}} \rVert}}\right) + \frac{\Delta_{1} \theta\left(\lambda_{2}\right)}{2 \lVert {\Delta} \rVert} \left(\frac{\Delta_{2} \mu}{\lVert {\Delta} \rVert} - \frac{\mu u_{{t2}}}{\lVert {u_{{t}} \rVert}}\right) & \text{for}\: \lVert {\Delta} \rVert > \epsilon \wedge \lVert {u_{{t}} \rVert} > \epsilon \\\frac{\mu r_{{t1}}}{4 \lVert {r_{{t}} \rVert}^{3}} \left(2 \lVert {r_{{t}} \rVert} r_{{t2}} \theta\left(\lambda^{{\prime}}_{1}\right) + 2 \lVert {r_{{t}} \rVert} r_{{t2}} \theta\left(\lambda^{{\prime}}_{2}\right) - \sqrt{2} r_{{t1}}^{2} \theta\left(\lambda^{{\prime}}_{1}\right) + \sqrt{2} r_{{t1}}^{2} \theta\left(\lambda^{{\prime}}_{2}\right) - \sqrt{2} r_{{t2}}^{2} \theta\left(\lambda^{{\prime}}_{1}\right) + \sqrt{2} r_{{t2}}^{2} \theta\left(\lambda^{{\prime}}_{2}\right) + 2 r_{{t2}} \max\left(0, \lambda^{{\prime}}_{1}\right) - 2 r_{{t2}} \max\left(0, \lambda^{{\prime}}_{2}\right)\right) & \text{for}\: \lVert {u_{{t}} \rVert} \leq \epsilon \wedge \lVert {\Delta} \rVert > \epsilon \wedge \lVert {r_{{t}} \rVert} > \epsilon \end{cases}
\end{equation}

\begin{equation}
  A_{21} = \begin{cases} \frac{1.0 \mu^{2} \left(\mu - 1\right) \theta\left(\lambda^{{\prime\prime}}_{1}\right)}{2 \mu^{2} - 2 \mu + 1} & \text{for}\: \lVert {r_{{t}} \rVert} \leq \epsilon \wedge \lVert {u_{{t}} \rVert} \leq \epsilon \\\frac{\mu r_{{t1}} r_{{t2}} \theta\left(\lambda^{{\prime}}_{1}\right)}{\lVert {r_{{t}} \rVert}^{2}} & \text{for}\: \lVert {\Delta} \rVert \leq \epsilon \wedge \lVert {r_{{t}} \rVert} > \epsilon \wedge \lVert {u_{{t}} \rVert} > \epsilon \\\frac{\Delta_{1} \Delta_{2} \mu \max\left(0, \lambda_{1}\right)}{2 \lVert {\Delta} \rVert^{3}} - \frac{\Delta_{1} \Delta_{2} \mu \max\left(0, \lambda_{2}\right)}{2 \lVert {\Delta} \rVert^{3}} - \frac{\Delta_{2} \theta\left(\lambda_{1}\right)}{2 \lVert {\Delta} \rVert} \left(- \frac{\Delta_{1} \mu}{\lVert {\Delta} \rVert} - \frac{\mu u_{{t1}}}{\lVert {u_{{t}} \rVert}}\right) + \frac{\Delta_{2} \theta\left(\lambda_{2}\right)}{2 \lVert {\Delta} \rVert} \left(\frac{\Delta_{1} \mu}{\lVert {\Delta} \rVert} - \frac{\mu u_{{t1}}}{\lVert {u_{{t}} \rVert}}\right) & \text{for}\: \lVert {\Delta} \rVert > \epsilon \wedge \lVert {u_{{t}} \rVert} > \epsilon \\\frac{0.25 \mu r_{{t2}}}{\lVert {r_{{t}} \rVert}^{3}} \left(2.0 \lVert {r_{{t}} \rVert} r_{{t1}} \theta\left(\lambda^{{\prime}}_{1}\right) + 2.0 \lVert {r_{{t}} \rVert} r_{{t1}} \theta\left(\lambda^{{\prime}}_{2}\right) - \sqrt{2} r_{{t1}}^{2} \theta\left(\lambda^{{\prime}}_{1}\right) + \sqrt{2} r_{{t1}}^{2} \theta\left(\lambda^{{\prime}}_{2}\right) + 2.0 r_{{t1}} \max\left(0, \lambda^{{\prime}}_{1}\right) - 2.0 r_{{t1}} \max\left(0, \lambda^{{\prime}}_{2}\right) - \sqrt{2} r_{{t2}}^{2} \theta\left(\lambda^{{\prime}}_{1}\right) + \sqrt{2} r_{{t2}}^{2} \theta\left(\lambda^{{\prime}}_{2}\right)\right) & \text{for}\: \lVert {u_{{t}} \rVert} \leq \epsilon \wedge \lVert {\Delta} \rVert > \epsilon \wedge \lVert {r_{{t}} \rVert} > \epsilon \end{cases}
\end{equation}

\begin{equation}
  A_{22} = \begin{cases} \frac{\mu^{3} \theta\left(\lambda^{{\prime\prime}}_{1}\right)}{2 \mu^{2} - 2 \mu + 1} & \text{for}\: \lVert {r_{{t}} \rVert} \leq \epsilon \wedge \lVert {u_{{t}} \rVert} \leq \epsilon \\\frac{\mu r_{{t2}}^{2} \theta\left(\lambda^{{\prime}}_{1}\right)}{\lVert {r_{{t}} \rVert}^{2}} & \text{for}\: \lVert {\Delta} \rVert \leq \epsilon \wedge \lVert {r_{{t}} \rVert} > \epsilon \wedge \lVert {u_{{t}} \rVert} > \epsilon \\- \frac{\Delta_{2} \theta\left(\lambda_{1}\right)}{2 \lVert {\Delta} \rVert} \left(- \frac{\Delta_{2} \mu}{\lVert {\Delta} \rVert} - \frac{\mu u_{{t2}}}{\lVert {u_{{t}} \rVert}}\right) + \frac{\Delta_{2} \theta\left(\lambda_{2}\right)}{2 \lVert {\Delta} \rVert} \left(\frac{\Delta_{2} \mu}{\lVert {\Delta} \rVert} - \frac{\mu u_{{t2}}}{\lVert {u_{{t}} \rVert}}\right) - \frac{\max\left(0, \lambda_{1}\right)}{2} \left(- \frac{\Delta_{2}^{2} \mu}{\lVert {\Delta} \rVert^{3}} + \frac{\mu}{\lVert {\Delta} \rVert}\right) - \frac{\max\left(0, \lambda_{2}\right)}{2} \left(\frac{\Delta_{2}^{2} \mu}{\lVert {\Delta} \rVert^{3}} - \frac{\mu}{\lVert {\Delta} \rVert}\right) & \text{for}\: \lVert {\Delta} \rVert > \epsilon \wedge \lVert {u_{{t}} \rVert} > \epsilon \\\frac{0.25 \mu}{\lVert {r_{{t}} \rVert}^{3}} \left(2.0 \lVert {r_{{t}} \rVert} r_{{t2}}^{2} \theta\left(\lambda^{{\prime}}_{1}\right) + 2.0 \lVert {r_{{t}} \rVert} r_{{t2}}^{2} \theta\left(\lambda^{{\prime}}_{2}\right) - \sqrt{2} r_{{t1}}^{2} r_{{t2}} \theta\left(\lambda^{{\prime}}_{1}\right) + \sqrt{2} r_{{t1}}^{2} r_{{t2}} \theta\left(\lambda^{{\prime}}_{2}\right) - 2.0 r_{{t1}}^{2} \max\left(0, \lambda^{{\prime}}_{1}\right) + 2.0 r_{{t1}}^{2} \max\left(0, \lambda^{{\prime}}_{2}\right) - \sqrt{2} r_{{t2}}^{3} \theta\left(\lambda^{{\prime}}_{1}\right) + \sqrt{2} r_{{t2}}^{3} \theta\left(\lambda^{{\prime}}_{2}\right)\right) & \text{for}\: \lVert {u_{{t}} \rVert} \leq \epsilon \wedge \lVert {\Delta} \rVert > \epsilon \wedge \lVert {r_{{t}} \rVert} > \epsilon \end{cases}
\end{equation}

\subsection{The Fischer Burmeister function}

\section{Conclusion}

Symbolic differentiation -> inefficient code

A better approach Symbolic code generation for the function, Automatic
differentiation of the smooth part + user provided values on nonsmooth part.



\bibliographystyle{plainnat}
\bibliography{biblio}

\end{document}
\endinput

%%% Local Variables: 
%%% mode: latex
%%% TeX-master: t
%%% TeX-engine: default-shell-escape 
%%% End: 
