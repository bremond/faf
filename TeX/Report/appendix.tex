\clearpage
\appendix

\section{Basics in Convex Analysis}
\label{Sec:Ann:ConvexAnalysis}
\begin{eqnarray}
y = P_K(x) & \Longleftrightarrow &
\begin{array}{ll}
  \min &\frac 1 2 (y-x)^\top (y-x ) \\
  \text {s.t. } & y \in K
\end{array}
\\
& \Longleftrightarrow & - (y-x) \in N_K(y) \\
& \Longleftrightarrow & (x-y)^\top(y-z) \geq 0, \forall z \in K 
\label{eq:tutu}
\end{eqnarray}

\begin{eqnarray}
-F(x) \in N_K(x) & \Longleftrightarrow & -\rho F(x)^\top(y-x) \geq 0, \forall y \in K\\
& \Longleftrightarrow &  (x-(x - \rho F(x))^\top(y-x) \geq 0, \forall y \in K \\
& \Longleftrightarrow &  x = P_K(x- \rho F(x)) \text{ thanks to (\ref{eq:tutu})} 
\end{eqnarray}

\begin{equation}
  \label{eq:sub-norm}
  \partial \|z\|_2 =
  \begin{cases}
      \Frac z {\|z\|}\\
      \{x, \|x\|=1 \}
    \end{cases}
\end{equation}
\subsection{Euclidean projection on the disk  in $\RR^2$.}
Let $D=\{x \in \RR^3, \|x\| \leq 1\}$. We have 
\begin{equation}
  \label{eq:projD}
  P_{D}(z) =\left\{
  \begin{array}{lcl}
    z &\text{if}& z \in D \\
    \Frac {z} {\|z\|}& \text{if}&z \notin D
    \end{array}\right. 
  \end{equation}
 and
\begin{equation}
  \label{eq:sub-projD}
  \partial P_{D}(z) =\left\{
  \begin{array}{lcl}
   I &\text{if }& z \in D \setminus \partial D \\
   I + (s-1)z z^\top , s\in [0,1]   &\text{if }& z \in  \partial D \\
   \Frac I {\|z\|} - \Frac{z z^\top}{\|z\|^3} &\text{if }& z \notin D  
   \end{array}\right. 
\end{equation}
\subsection{Euclidean projection on the second order cone of $\RR^3$.}

Let $K=\{x = [x_\n x_\t]^T \in \RR^3, x_\n \in \RR, \|x_{\t}\| \leq \mu x_{\n}\}$. We have 
\begin{equation}
  \label{eq:proj}
  P_{K}(z) =\left\{
  \begin{array}{lcl}
    z &\text{if}& z \in K \\
    0 &\text{if}& -z \in K^* \\
    \Frac 1 {1+\mu^2}(z_\n + \mu \|z_\t\| )
    \left[\begin{array}{c}
      1 \\
      \mu \Frac {z_\t} {\|z_\t\|}
      \end{array}\right]&\text{if}& z \notin K \text{ and } -z \notin K^*
    \end{array}\right. 
  \end{equation}
The computation of the subdifferential are given as follows
\begin{itemize}
\item if $z\in K\setminus \partial K$, $\partial_z P_K(z) =I$,
\item if $-z \in K^*\setminus \partial K^*$, $\partial_z P_K(z) =0$,
\item if $z \notin K$ and $-z \notin K^*$ and, $\partial_z P_K(z) =0$, we get
\begin{equation}
  \label{eq:sub-proj_n}
  \partial_{z_\n} P_{K}(z) =
  % \left\{\begin{array}{lcl}
  \Frac 1 {1+\mu^2}
  \left[
    \begin{array}{c}
      1 \\
      \mu z_\t
    \end{array}\right] 
  % \end{array}\right. 
\end{equation}
and
\begin{equation}
  \label{eq:sub-proj_t1}
  \partial_{z_\t} [P_{K}(z)]_\n =
  % \left\{\begin{array}{lcl}
  \Frac \mu {1+\mu^2} \Frac {z_\t} {\|z_\t\|}
  % \end{array}\right. 
\end{equation}
\begin{equation}
  \label{eq:sub-proj_2}
  \partial_{z_\t} [P_{K}(z)]_\t =
  % \left\{\begin{array}{lcl}
  \Frac {\mu} {(1+\mu^2)} \left[\mu \Frac {z_\t} {\|z_\t\|} \Frac {z^\top_\t} {\|z_\t\|}+ (z_\n + \mu \|z_\t\|) (\Frac{I_2}{\|z_\t\|} - \Frac{z_\t z_\t^\top}{\|z_\t\|^3} )\right]
  % \end{array}\right. 
\end{equation}
that is
\begin{equation}
  \label{eq:sub-proj_2}
  \partial_{z_\t} [P_{K}(z)]_\t =
  % \left\{\begin{array}{lcl}
  \Frac {\mu} {(1+\mu^2)\|z_\t\|} \left[ (z_\n + \mu \|z_\t\|) \,I_2   +  z_\n  \Frac{z_\t z_\t^\top}{\|z_\t\|^2} )\right]
  % \end{array}\right. 
\end{equation}
\begin{ndrva}
  to be checked carefully
\end{ndrva}

\end{itemize}





In~\cite{Hayashi.ea2005}, the computation of the Clarke subdifferential of the projection operator is also done by inspecting the different cases using the spectral decomposition 
\begin{equation}
  \label{eq:Jordan-projection-Jacobian}
  \partial P_K(x) =
  \left\{
    \begin{array}{cl}
      I & \quad (\lambda_1>0, \lambda_2 >0 ) \\
      \Frac{\lambda_2}{\lambda_1+\lambda_2} I +Z & \quad (\lambda_1< 0, \lambda_2 >0 ) \\
      0 & \quad (\lambda_1< 0, \lambda_2 <0 ) \\
      \co{I, I+Z} & \quad (\lambda_1=0, \lambda_2 >0 ) \\
      \co{0, Z} & \quad (\lambda_1<0, \lambda_2 =0 ) \\
      \co{0,  I, Z}  & \quad (\lambda_1=0, \lambda_2 =0 ) 
    \end{array}
  \right.
\end{equation}
A simple verification shows that the previous computation is an element of the subdifferential.
\begin{ndrva}
  to be checked carefully
\end{ndrva}



\section{Computation of components of subgradient of $F_{\vitwo}^\nat$}
\label{Sec:Phi-natural-typeII}
 For one contact, a possible computation of the remaining parts in $\Phi(u, r)$  is given by
\begin{equation}
  \label{eq:Phi-natural-typeII-rr}
   \Phi_{r u}(u,r)  = 
  \left\{
    \begin{array}{lcl}
      0  & \text { if } & r- \rho(u+g(u)) \in K    \\ \\
      I - \partial_{r}[P_K(r-\rho(u+g(u)))]  & \text { if } &r- \rho(u+g(u))   \notin K
    \end{array}\right.
\end{equation}

\begin{equation}
  \label{eq:Phi-natural-typeII-ru}
  \Phi_{r u}(u,r) = 
  \left\{
    \begin{array}{lcl}
      \rho \left(I +
      \left[\begin{array}{ccc}
          0 & 0 & 0 \\
          \Frac{u_\t}{\|u_\t\|} & 0 & 0 \\
        \end{array}\right]
      \right)  & \text { if } &
      \begin{cases}
        r- \rho(u+g(u)) \in K \\ u_\t \neq 0
      \end{cases}
      \\ \\
      \rho\left(I +  \left[\begin{array}{ccc}
          0 & 0 & 0 \\
          s & 0 & 0 \\
        \end{array}\right]\right), s \in \RR^2 , \|s\|=1  & \text { if } &
    \begin{cases}
      r- \rho(u+g(u)) \in K \\ u_\t = 0
    \end{cases}
    \\ \\
      I        +\rho \left(I +  \left[\begin{array}{ccc}
          0 & 0 & 0 \\
          \Frac{u_\t}{\|u_\t\|} & 0 & 0 \\
        \end{array}\right]\right)    \partial_{u}[P_K(r-\rho(u+g(u)))] & \text { if } &r- \rho(u+g(u))   \notin K
    \end{array}\right.
\end{equation}
The computation of an element of $\partial P_K$ is given in Appendix~\ref{Sec:Ann:ConvexAnalysis}.


%%% Local Variables: 
%%% mode: latex
%%% TeX-master: "rr"
%%% End: 
