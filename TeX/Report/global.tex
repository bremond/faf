% This file contains all the tex code relative to the global FC that were deleted from the original paper
% This material should serve as a basis for a paper dealing with those issues.


\subsection{Frictional contact discrete problems}

Let us now introduce two basic frictional contact problems defined in terms of complementarity over cones.
\begin{problem}[Discrete frictional contact problem]\label{prob:I}
  Given
  \begin{itemize}
    \item a symmetric positive definite matrix ${M} \in \nbR^{n \times n}$,
    \item a vector $ {f} \in \nbR^n$,
    \item a matrix  ${H} \in \nbR^{n \times m}$,
    \item a vector $w \in \RR^{m}$,
    \item a vector of coefficients of friction $\mu \in \RR^{n_c}$,
  \end{itemize}
find two vectors $ {v} \in \nbR^n$ and $r\in \RR^m$, denoted by $\mathrm{FC/I}(M,H,f,w,\mu)$  such that
\begin{equation}\label{eq:soccp1}
  \begin{cases}
    M v = {H} {r} + {f} \\[2mm]
    u = H^\top v + w \\[2mm]
    \hat u = u + g(u) \\[2mm]
    K^\star \ni {\hat u} \perp r \in K
  \end{cases}
\end{equation}
with $g(u) = [[\mu^\alpha  \|u^\alpha_\t\| {\sf N}^\alpha]^\top, \alpha = 1\ldots n_c]^\top$. 
\qed
\end{problem}


Problem~\ref{prob:II} is also often referenced as the \emph{condensed} or \emph{reduced} problem. The two problems are closely related. Indeed, if we consider the inverse of the matrix $M$ we obtain an explicit equation for $v$ in Problem~\ref{prob:I}
\begin{equation}
  \label{eq:vv}
  v = M^{-1}\,(Hr +f).
\end{equation}
Substituting~\eqref{eq:vv} in the first line of~\eqref{eq:soccp1}, we get
\begin{equation}
  \label{eq:vv-1}
  u = H^\top M^{-1} H r + H^\top M^{-1} f +w.
\end{equation}
The matrix $W$, often called the \emph{Delassus matrix}, is easily identified as
\begin{equation}
  \label{eq:Delassus}
  W = H^\top M^{-1} H 
\end{equation}
and the vector $q$ as
\begin{equation}
  \label{eq:qq}
  q = H^\top M^{-1} f + w.
\end{equation}

\paragraph{Comments} The variable $\hat u$ does not explicitly appear as a variable of the problem since the mapping $u \mapsto u+g(u)$ is a one--to--one mapping.

\tb{Even if the matrix $M$ is invertible, the problem $\mathrm{FC/I}(M,H,f,w,\mu)$ and $\mathrm{FC/II}(W,q,\mu)$ are different from the point of view of their structure. The problem $\mathrm{FC/I}(M,H,f,w,\mu)$ enjoys good properties of sparsity in most of the applications. Either if $M$ comes from multi--body rigid systems or finite--element applications, it is sparse. It is also the same for $H$ since a contact relation involves at best two bodies. When we compute $W$, this sparsity may be lost for instance for systems that comes from finite--element applications, but the system is smaller in terms of number of unknowns. In~$\mathrm{FC/II}(W,q,\mu)$, the structure of $q$ has to be remarked from~\eqref{eq:qq}. Again, if $w=0$ or $w \in \mathrm{im}(H^\top)$, $q$ belongs to the range of $H^\top$ and we will see that it has some consequence on the existence of solutions in the sequel.}

\subsection{Variational Inequalities (VI) formulations}

For Problem~\ref{prob:I}, let us start by rewriting the problem~\eqref{eq:soccp1} as a normal cone inclusion
\begin{equation}
  \label{eq:soccp1-bis}
  \left\{\begin{array}{l}
    M v-H r-f=0, \\[2mm]
   -(H^\top v + w + g(H^\top v + w)) \in N_K(r).
 \end{array}\right.
\end{equation}
The problem~\eqref{eq:soccp1-bis} is written as an inclusion into the normal cone to $\bar K = \RR^n  \times K\subset \RR^{m+n}$ as
\begin{equation}
  \label{eq:soccp1-ter}
  - \left[\begin{array}{l}
    M v-H r-f \\
    H^\top v+w  + g(H^\top v +w)
 \end{array}\right]    \in N_{\bar K}\left(
\left[\begin{array}{l}
  v \\r
\end{array}\right]
\right).
\end{equation}
Finally, the VI formulation of Problem~\ref{prob:I} is given by
\begin{equation}
  \label{eq:vi-I-bis}
  \left[\begin{array}{l}
    M v-H r-f \\
    H^\top v+w  + g(H^\top v +w)
  \end{array}\right] \left(
  \left[\begin{array}{l}
      p  \\ s
    \end{array}\right] - \left[\begin{array}{l}
      v  \\ r
    \end{array}\right]
\right) \geq 0,\quad \text{ for all } s \in K,  p \in \RR^n.
\end{equation}
Let us introduce a convenient notation for this formulation as $\mathrm{VI}(F_{\vione},X_{\vione})$ with
\begin{equation}
  \label{eq:vi-I}
  F_{\vione}(v,r) \coloneqq \left[\begin{array}{l}
    M v-H r-f \\
    H^\top v+w  + g(H^\top v +w)
  \end{array}\right],\quad \text{ and } X_{\vione} = \bar K = \RR^n \times K\subset \RR^{n+m}.
\end{equation}

It is also therefore easy to conclude to the existence of a unique solution for  $\mathrm{VI}(F_{\vione},X_{\vione})$.

 we obtain for Problem~\ref{prob:I} under the VI form~\eqref{eq:vi-I},
\begin{equation}
  \label{eq:natural-I}
  F_\vione^\nat(v,r) \coloneqq \left[
  \begin{array}{l}
    \rho (Mv - Hr + f) \\
    r - P_{K}\left(r  - \rho (H^\top v +w  + g(H^\top v +w))\right)
  \end{array}\right] = 0,
\end{equation}
and  for 


\subsection{Nonsmooth Equations}

Problem~\ref{prob:I} under the VI form~\eqref{eq:vi-I},
\begin{equation}
  \label{eq:natural-typeI}
  F_{\vione,D}^\nat(v,r) \coloneqq \left[
  \begin{array}{l}
    D_1^{-1} (Mv - Hr + f) \\ 
    r - P_{K}\left(r  - D_2^{-1} (H^\top v + w  + g(H^\top v + w))\right)
  \end{array}\right] = 0
\end{equation}
with $D_1\in \RR^{n\times n}$ and $D_2\in \RR^{m\times m}$.
For

\paragraph{Jean--Moreau's and Alart-Curnier's functions}

Problem~\ref{prob:I} is then reformulated as
\begin{equation}
  \label{eq:MJ-I}
  F_{\mjone}(v,r) \coloneqq \left[
    \begin{array}{c}
    Mv - Hr - f \\
    r_\n - P_{\RR^{n_c}_+}(r_\n - \rho_\n (H^\top v + w)_\n) \\
    r_\t - P_{D(\mu, (r_{\n})_+)}(r_\t - \rho_\t (H^\top v + w)_\t   ) 
  \end{array}\right] =0
\end{equation}
and  


Problem~\ref{prob:I} is then reformulated as
\begin{equation}
  \label{eq:AC-I}
  F_{\acone}(v,r) \coloneqq \left[
    \begin{array}{c}
    Mv - Hr - f \\
    r_\n - P_{\RR^{n_c}_+}(r_\n - \rho_\n (H^\top v +w)_\n) \\
    r_\t - P_{D(\mu, (r_\n - \rho_\n u_\n)_+)}(r_\t - \rho_\t (H^\top v +w)_\t   )
  \end{array}\right] =0
\end{equation}
and  

\subsection{Optimization problems}
\paragraph{The alternating optimization approach}

Problem~\ref{prob:I} is treated in the same manner by splitting the matrix $H$ and the vector $w_\n$ such that
\begin{equation}
  \label{eq:H-split}
  u = H^\top v +w \Longleftrightarrow  \left[\begin{array}{c}
    u_\n \\ u_\t
  \end{array}\right]
 =
  \left[\begin{array}{c}
      H_{\n} \\
      H_{\t} \\    
    \end{array}\right]^\top   \, v+
 \left[\begin{array}{c}
    w_\n \\w_\t
  \end{array}\right].
\end{equation}
We get for the normal part
\begin{equation}\label{eq:soccp1-normal}
  \begin{cases}
    M v = {H_\n} {r_\n} + {\tilde f_\n} \\[2mm]
    u_\n = H_\n^\top v + w_\n \\[2mm]
    0 \leq { u_\n} \perp r _\n \geq 0
  \end{cases}
\end{equation}
for  given value of $\tilde r_\t$ and $\tilde f_\n = f + H_\t r_\t$. Since $M$ is a symmetric positive definite matrix, an equivalent convex optimization can be written
\begin{equation}
  \label{eq:soccp1-normal-min}
    \begin{cases}
    \min\, \frac 1 2 v^\top M v + v^\top \tilde f_\n  \\[2mm]
     \begin{array}{ll}
    s.t. & H_\n^\top v + w_\n \geq 0
  \end{array}
  \end{cases}
\end{equation}
and $u_\n$ is obtained by $ u_n = H_\n^\top v + w_\n $ and $r_\n$ is the Lagrange multiplier associated with the constraints. For the tangent part, the situation is a slightly more complicated since we get
\begin{equation}\label{eq:soccp1-tangent}
  \begin{cases}
    M v = {H_\t} {r_\t} + {\tilde f_\t} \\[2mm]
    u_\t = H_\t^\top v + w_\t \\[2mm]
    -u_\t \in N_{D(\mu,\tilde r_{\n})}(r_\t),
  \end{cases}
\end{equation}
for  given value of  $\tilde r_\n$ and $\tilde f_\t = f + H_\n \tilde r_\n$. It is therefore difficult to associate an optimization which differs from~\eqref{eq:AO-4}.  

\paragraph{The convex SOCP}

Problem~\ref{prob:I} may be also reformulated in this form.
\begin{equation}\label{eq:ACLM-4}
  \begin{cases}
    s = g(u) \\[2mm]
    \left\{
      \begin{array}{ll}
        \min\,&\Frac 1 2 v^\top M v -  f^\top v  \\
        \text{s.t. } & H v  + w + s  \in K^\star.
      \end{array}
    \right.
  \end{cases}
\end{equation} 
In this latter formulation, the optimization sub--problem appears as a strictly convex optimization problem ($M$ is assumed to be semi--definite positive) under (possibly redundant) cone constraints. If this sub--problem is feasible, the sub--problem admits a unique solution. 

\subsection{Basic fixed point  and projection methods for VI}

Applying the Algorithm~\ref{Algo:FP-vi} to the VI formulation~\eqref{eq:vi-I} leads to the following iteration
\begin{equation}
  \label{eq:FP-vi-I}
  \begin{cases}
    v_{k+1} \leftarrow  v_k - D_1^{-1}( M v_{k} - H r_k +f) \\
    u_{k+1} \leftarrow  u_k - D_2^{-1}( u_k - H^\top v_{k} - w) \\ 
    r_{k+1} \leftarrow  P_{K,D_3}(r_k - D_3^{-1}(u_k +  g(u_k))).
  \end{cases}
\end{equation}
Due to the fact that $u$ is explicitly given as a function of $v$, $D_2$ is chosen as the identity matrix and $u_{k+1}$ plays the role of $u_k$ in the last equation. The following iteration is obtained
\begin{equation}
  \label{eq:FP-vi-Ibis}
  \begin{cases}
    v_{k+1} \leftarrow  v_k - D_1^{-1}( M v_{k} - H r_k +f) \\
    r_{k+1} \leftarrow  P_{K,D_3}(r_k - D_3^{-1}(H^\top v_k + w +  g(H^\top v_k+ w ))).
  \end{cases}
\end{equation}

\subsection{Application to  the discrete frictional contact problem}
\subsubsection{Nonsmooth newton based on the natural map}

Let us consider the natural map $F_{\vione}^\nat$ in~\eqref{eq:natural-I} that enables to write Problem~\ref{prob:I} in a nonsmooth equation.  Algorithm~\ref{Algo:NSN} may be also applied or $z =(v,r)$ with
\begin{equation}
  \label{eq:phiphi_bis}
  \begin{cases}
    \Phi(v,r) \in \partial_{v,r} F_{\vione}^\nat(r)
  \end{cases}
\end{equation}


