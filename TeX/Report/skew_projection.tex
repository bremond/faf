\documentclass[a4]{article}

\begin{document}
Let us consider the definition of the skew projection operator on the second order cone $K$  as 
\begin{equation}
  P_{K,D} (x)  = \mbox{argmin}_{z \in K} \frac 1 2 (z-x)^T D (z-x)
\end{equation}
with $D=D^T>0$.

\textbf{Question : How to explicitly compute a (generalized) gradient of  $P_{K,D} (x)$ when $x \notin K$ ?}

Clearly, if $x \in K \setminus \partial K$, this is obvious.  If $D = \rho I$ this is also trivial. But in the general case, it seems a little more tricky. any idea ?



%  The KKT conditions are given by
%  \begin{equation}
%    \label{eq:KKT}
%   -   D(y-x) \in N_{K}(y)
%  \end{equation}
% or equivalently, if K is a cone
% \begin{equation}
%   \label{eq:CP}
%   \begin{array}[c]{c}
%     K^* \ni D(y-x) \perp z \in \in K
%   \end{array}
%  \end{equation}




\end{document}


%%% Local Variables: 
%%% mode: latex
%%% TeX-master: "skew_projection"
%%% End: 
