

%% Symbole de fraction
\newcommand{\Frac}[2]{{\displaystyle \frac{\displaystyle #1}{\displaystyle #2}}}
\newcommand{\Prac}[2]{\displaystyle \genfrac{(}{)}{}{}{\displaystyle #1}{\displaystyle #2}}
\newcommand{\Crac}[2]{\displaystyle \genfrac{[}{]}{}{}{\displaystyle #1}{\displaystyle #2}}

\newcommand{\norme}[1]{\|#1\|}


\newcommand{\HRule}{\rule{\linewidth}{1mm}}

% Fonction mathematiques

\newcommand{\transposee}[1]{{\vphantom{#1}}^{\text{\tiny{\textsf T}}}{#1}}
\newcommand{\argmin}{\mathop{\mathrm{argmin}}}
\newcommand{\argminn}{\mathop{\mathrm{argmin}}}
\newcommand{\lexicomin}{\mathop{\mathrm{lexicomin}}}
%\newcommand{\arg}{\mathop{\mathrm{arg}}}



\DeclareMathOperator{\rot}{rot}
\DeclareMathOperator{\sh}{sh}
\DeclareMathOperator{\ch}{ch}
%\DeclareMathOperator{\th}{th}
\DeclareMathOperator{\arcsh}{arcsh}
\DeclareMathOperator{\argth}{argth}
\DeclareMathOperator{\sign}{sign}


%%The Principal Value Integral symbol
\def\Xint#1{\mathchoice
   {\XXint\displaystyle\textstyle{#1}}%
   {\XXint\textstyle\scriptstyle{#1}}%
   {\XXint\scriptstyle\scriptscriptstyle{#1}}%
   {\XXint\scriptscriptstyle\scriptscriptstyle{#1}}%
   \!\int}
\def\XXint#1#2#3{{\setbox0=\hbox{$#1{#2#3}{\int}$}
     \vcenter{\hbox{$#2#3$}}\kern-.5\wd0}}
\def\ddashint{\Xint=}
\def\dashint{\Xint-}



% macro pour les symbols d'ensemble
%\nbOne
\def\nbOne{{\mathchoice{\rm 1\mskip-4mu l}{\rm 1\mskip-4mu l} {\rm 1 \mskip-4.5mu l}{\rm 1\mskip-5mu l}}}
%
%%  Les ensembles de nombres  C. Fiorio (fiorio@math.tu-berlin.de) 
%
\def\nbR{\ensuremath{\mathrm{I\!R}}} % IR
\def\nbN{\ensuremath{\mathrm{I\!N}}} % IN
\def\nbF{\ensuremath{\mathrm{I\!F}}} % IF
\def\nbH{\ensuremath{\mathrm{I\!H}}} % IH
\def\nbK{\ensuremath{\mathrm{I\!K}}} % IK
\def\nbL{\ensuremath{\mathrm{I\!L}}} % IL
\def\nbM{\ensuremath{\mathrm{I\!M}}} % IM
\def\nbP{\ensuremath{\mathrm{I\!P}}} % IP
%
% \nbOne : 1I : symbol one
\def\nbOne{{\mathchoice {\rm 1\mskip-4mu l} {\rm 1\mskip-4mu l}
{\rm 1\mskip-4.5mu l} {\rm 1\mskip-5mu l}}}
%
% \nbC   :  Nombres Complexes
\def\nbC{{\mathchoice {\setbox0=\hbox{$\displaystyle\rm C$}%
\hbox{\hbox to0pt{\kern0.4\wd0\vrule height0.9\ht0\hss}\box0}}
{\setbox0=\hbox{$\textstyle\rm C$}\hbox{\hbox
to0pt{\kern0.4\wd0\vrule height0.9\ht0\hss}\box0}}
{\setbox0=\hbox{$\scriptstyle\rm C$}\hbox{\hbox
to0pt{\kern0.4\wd0\vrule height0.9\ht0\hss}\box0}}
{\setbox0=\hbox{$\scriptscriptstyle\rm C$}\hbox{\hbox
to0pt{\kern0.4\wd0\vrule height0.9\ht0\hss}\box0}}}}
%
% \nbQ   : Nombres Rationnels Q
\def\nbQ{{\mathchoice {\setbox0=\hbox{$\displaystyle\rm
Q$}\hbox{\raise
0.15\ht0\hbox to0pt{\kern0.4\wd0\vrule height0.8\ht0\hss}\box0}}
{\setbox0=\hbox{$\textstyle\rm Q$}\hbox{\raise
0.15\ht0\hbox to0pt{\kern0.4\wd0\vrule height0.8\ht0\hss}\box0}}
{\setbox0=\hbox{$\scriptstyle\rm Q$}\hbox{\raise
0.15\ht0\hbox to0pt{\kern0.4\wd0\vrule height0.7\ht0\hss}\box0}}
{\setbox0=\hbox{$\scriptscriptstyle\rm Q$}\hbox{\raise
0.15\ht0\hbox to0pt{\kern0.4\wd0\vrule height0.7\ht0\hss}\box0}}}}
%
% \nbT   : T
\def\nbT{{\mathchoice {\setbox0=\hbox{$\displaystyle\rm
T$}\hbox{\hbox to0pt{\kern0.3\wd0\vrule height0.9\ht0\hss}\box0}}
{\setbox0=\hbox{$\textstyle\rm T$}\hbox{\hbox
to0pt{\kern0.3\wd0\vrule height0.9\ht0\hss}\box0}}
{\setbox0=\hbox{$\scriptstyle\rm T$}\hbox{\hbox
to0pt{\kern0.3\wd0\vrule height0.9\ht0\hss}\box0}}
{\setbox0=\hbox{$\scriptscriptstyle\rm T$}\hbox{\hbox
to0pt{\kern0.3\wd0\vrule height0.9\ht0\hss}\box0}}}}
%
% \nbS   : S
\def\nbS{{\mathchoice
{\setbox0=\hbox{$\displaystyle     \rm S$}\hbox{\raise0.5\ht0%
\hbox to0pt{\kern0.35\wd0\vrule height0.45\ht0\hss}\hbox
to0pt{\kern0.55\wd0\vrule height0.5\ht0\hss}\box0}}
{\setbox0=\hbox{$\textstyle        \rm S$}\hbox{\raise0.5\ht0%
\hbox to0pt{\kern0.35\wd0\vrule height0.45\ht0\hss}\hbox
to0pt{\kern0.55\wd0\vrule height0.5\ht0\hss}\box0}}
{\setbox0=\hbox{$\scriptstyle      \rm S$}\hbox{\raise0.5\ht0%
\hboxto0pt{\kern0.35\wd0\vrule height0.45\ht0\hss}\raise0.05\ht0%
\hbox to0pt{\kern0.5\wd0\vrule height0.45\ht0\hss}\box0}}
{\setbox0=\hbox{$\scriptscriptstyle\rm S$}\hbox{\raise0.5\ht0%
\hboxto0pt{\kern0.4\wd0\vrule height0.45\ht0\hss}\raise0.05\ht0%
\hbox to0pt{\kern0.55\wd0\vrule height0.45\ht0\hss}\box0}}}}
%
% \nbZ   : Entiers Relatifs Z
\def\nbZ{{\mathchoice {\hbox{$\sf\textstyle Z\kern-0.4em Z$}}
{\hbox{$\sf\textstyle Z\kern-0.4em Z$}}
{\hbox{$\sf\scriptstyle Z\kern-0.3em Z$}}
{\hbox{$\sf\scriptscriptstyle Z\kern-0.2em Z$}}}}
%%%% fin macro %%%%



\newcommand{\putidx}[1]{\index{#1}\textit{#1}}


%\definecolor{darkgray}{gray}{.25}
\definecolor{gray}{gray}{.5}
\definecolor{lightgray}{gray}{.75}
%\definecolor{gradbegin}{rgb}{0,1,1}
%\definecolor{gradend}{rgb}{0,.1,.95}
%\newcommand{\newtexte}[1]{\textcolor{darkgray} {#1}}
\newcommand{\newtexte}[1]{{#1}}% macro pour les varibales favorites
% normal tangent
\def\n{{\hbox{\tiny{N}}}}
\def\t{{\hbox{\tiny{T}}}}
\def\ss{{\hbox{\tiny{S}}}}
\def\nt{\hbox{\tiny{NT}}}
\def\nsf{\hbox{\tiny{\textsf N}}}
\def\tsf{\hbox{\tiny{\textsf T}}}
\def\sigman{\sigma_{\n}}
\def\sigmat{\sigma_{\t}}
\def\sigmant{\sigma_{\nt}}
\def\epsn{\epsilon_{\n}}
\def\epst{\epsilon_{\t}}
\def\epsnt{\epsilon_{\nt}}
\def\eps{\epsilon}
\def\veps{\varepsilon}
\def\sig{\sigma}
\def\Rn{R_{\n}}
\def\Rt{R_{\t}}
\def\cn{c_{\n}}
\def\Cn{C_{\n}}
\def\ct{c_{\t}}
\def\Ct{C_{\t}}
\def\un{u_{\n}}
\def\ut{\buu_{\t}}
\def\uut{u_{\t}}
\def\unc{u_{\n}^c}
\def\utc{\buu_{\t}^c}
\def\vn{v_{\n}}
\def\vt{v_{\t}}
\def\rr{\hbox{\tiny{\textsf R}}}
\def\irr{\hbox{\tiny{\textsf{IR}}}}
\def\rn{r_{\n}}
\def\rt{\brr_{\t}}
\def\rnc{r_{\n}^c}
\def\rtc{\brr_{\t}^c}
\def\trn{\Tilde{r}_{\n}}
\def\trt{\Tilde{\brr}_{\t}}
\def\tr{\Tilde{\brr}}
\def\tv{\Tilde{\bvv}}
\def\vn{v_{\n}}
\def\vt{\bvv_{\t}}
\def\adh{\mathsf{adh}}
\def\adj{\hbox{\tiny{\textsf{adj}}}}
\def\adjc{\hbox{\tiny{\textsf{adjC}}}}
\def\adja{\hbox{\tiny{\textsf{adjA}}}}
\def\cc{\hbox{\tiny{\textsf C}}}
\def\ca{\hbox{\tiny{\textsf A}}}

% domaines et frontieres
\def\om{\Omega}
\def\oma{\Omega^{\alpha}}
\def\omu{\Omega^1\cup \Omega^2}
\def\gc{\Gamma_c}
\def\omt{\omu \cup \gc}
% derivee partielle et gradient et divergence
\def\p{\partial}
\def\grad{\nabla}
\def\div{\mathop{\rm div}\nolimits}
%

%\DeclareTextSymbol{\deg}{T1}{6}
%\def\degre{\mathdegree}
%\newcommand{\degre}{\mathdegree}

\def\etc{\textit{etc}\ldots}
\newcommand{\mdegre}{\hbox{\text{\degre}}}

%\def\nscd{\textsf{\bfseries NSCD}}
%\def\nscd{\textsf{NSCD}}
\newcommand{\nscd}{\textsf{NSCD}}
%\Pisymbol{psy}{212} ou encore \Pisymbol{psy}{228}




%----------------------------------------------------------------------
%             Des chiffres avec des ronds autour
%----------------------------------------------------------------------
\def\nombrecercle#1{\def\taille{0.3}
                \put(0,0){#1}
                \put(0.08,0.08){\circle{\taille}}}



\def\ae#1{\stackrel{\mbox{\scriptsize a.e.}}{#1}}
\def\argmin{\mathop{\rm argmin}}
\def\eqref#1{{\rm (\ref{#1})\/}}
\def\indicfon{\mathord{\rm i}}       %indicator function
\def\p{\mathord{\rm proj}}
\def\N{\mathord{\rm N}}
% \def\prosca#1#2{#1\cdot#2}
\def\prosca#1#2{\langle #1,#2\rangle}
\def\qedtext{\mbox{}\hfill$\Box$}
\def\qedmath{\eqno\Box}

\def\s{{$\mathcal{S}$}}
\def\somme{\mathop{\textstyle\sum}}
\def\somme{\mathop{\textstyle\sum}}
\def\submoins{_{\scriptscriptstyle-}}
\def\subplus{_{\scriptscriptstyle+}}
\def\T{\mathord{\rm T}}

%----------------------------------------------------------------------
%             Macro M Jean 
%----------------------------------------------------------------------

\def\Real{\mbox{I\hspace{-.15em}R}}
\def\Integer{\mbox{I\hspace{-.15em}N}}
\def\Bunit{\mbox{I\hspace{-.15em}B}}
\def\real{\mbox{\scriptsize I\hspace{-.15em}R}}
\def\bunit{\mbox{\scriptsize I\hspace{-.15em}B}}
\def\IL{\mbox{\scriptsize I\hspace{-.15em}L}}
\def\Indic{\mbox{\large $\psi$}}
\def\bfxi{\mbox{$\xi$ \hspace{-1.1em} $\xi$}}
%\def\bfXi{\mbox{$\Xi$ \hspace{-1.1em} $\Xi$}}
\def\RunR{\mathcal R}
\def\RunRN{\mathcal R_{N}}
\def\RunRT{\mathcal R_{T}}
\def\RunS{\mathcal S}
\def\RunSN{\mathcal S_{N}}
\def\RunST{\mathcal S_{T}}
\def\RunU{\mathcal U}
\def\RunUN{\mathcal U_{N}}
\def\RunUT{\mathcal U_{T}}
\def\RunUP{\mathcal U'}
\def\RunUPN{\mathcal U'_{N}}
\def\RunUPT{\mathcal U'_{T}}
\def\RunJ{\mathcal J}
\def\RunW{\mathcal W}
\def\RunF{f}
\def\RunFa{f_{1}}
\def\RunFb{f_{2}}
\def\RunFP{f'}
\def\RunV{v}
\def\RunVP{v'}
\def\EspF{\mathcal F}
\def\EspV{\mathcal V}

\def\N{\mbox{I\hspace{ -.15em}N}}
\def\Z{\mbox{Z\hspace{ -.3em}Z}}
\def\Q{\mbox{l\hspace{ -.47em}Q}}
\def\R{\mbox{l\hspace{ -.15em}R}}
\def\F{\mbox{l\hspace{ -.15em}F}}
\def\E{\mbox{l\hspace{ -.15em}E}}
\def\LMGC90{{\small \it LMGC90 }}
\def\NSCD{{\small \it NSCD }}
\def\CHIC{{\small \it CHIC }}
\def\half{{\frac{_{1}}{^{2}}}}
\def\12T{{\frac{_{1}}{^{2T}}}}

\def\geq{\geqslant}
\def\leq{\leqslant}
\def\ge{\geqslant}
\def\le{\leqslant}


\begingroup
\count0=\time \divide\count0by60 % Hour
\count2=\count0 \multiply\count2by-60 \advance\count2by\time
% Min
\def\2#1{\ifnum#1<10 0\fi\the#1}
\xdef\isodayandtime{\the\year-\2\month-\2\day\space\2{\count0}:%
\2{\count2}}
\endgroup

%---------------------------------------------------------------------
%             Redaction note environnement B. Brogliato
%----------------------------------------------------------------------
\makeatletter

{\newtheorem{ndr1bb}{\textbf{\textsc{Redaction note B.B.}}}[section]}

\newenvironment{ndrbb}%
{%
\noindent\begin{ndr1bb}\hrule\vspace{1em}%
\ttfamily\small
}%
{%
\begin{flushright}%
%\vspace{-1.5em}\ding{111}
\end{flushright}%
\vspace{-1.5em}\hrule
\end{ndr1bb}%
}
%----------------------------------------------------------------------
%             Redaction note environnement V.ACARY
%----------------------------------------------------------------------
% Faut etre fou pour s'amuser a pondre des notes pareilles

{\newtheorem{ndr1va}{\textbf{\textsc{\footnotesize Redaction note V.A.}}}[section]}

\newenvironment{ndrva}%
{%
\noindent\begin{ndr1va}\hrule\vspace{1em}%
\ttfamily\small \  \\
\noindent}%
{%
$ $ \\
\hrule
\end{ndr1va}%
}
\makeatother








% ----------------DEFINITIONS-----------------
% 

 \def\II{\mathop{{\rm I}\mskip-3.0mu{\rm I}}\nolimits}




% -----------------------------------
 \def\c{\mathop{{\rm 1}\mskip-10.0mu{\rm C}}\nolimits}
 \def\C{\mathop{{\rm 1}\mskip-10.0mu{\rm C}}\nolimits}
 \def\ZZ{\mathaccent23Z}
% 

\newcommand{\ie}{{\textit{i.e.}}}


%\def\sgn{\mbox{\rm sgn}}
\DeclareMathOperator{\sgn}{sgn}
\DeclareMathOperator{\proj}{proj}
\DeclareMathOperator{\prox}{prox}

%\newcommand{\RR}{\mbox{\rm $I\!\!R$}}
%\newcommand{\NN}{\mbox{\rm $I\!\!N$}}


\def\RR{\nbR}
\def\NN{\nbN}

% ---------------- MMC -----------------
% 

\newcommand{\contract}{{\,:\,}}

\newcommand{\scontract}{{\,{\Bar\otimes}\,}}
\newcommand{\tcontract}{{\,{\Bar{\Bar{\Bar\otimes}}}\,}}


\newcommand{\DP}[2]{\displaystyle \frac{\partial {#1}}{\partial {#2}}}

\newtheorem{definition}{Definition}
\newtheorem{proposition}{Proposition}
\newtheorem{lemma}{Lemma}

\newtheorem{claim}{Claim}
\newtheorem{remark}{Remark}
\newtheorem{assumption}{Assumption}
\newtheorem{example}{Example}
\newtheorem{conjecture}{Conjecture}
\newtheorem{corollary}{Corollary}
\newtheorem{OP}{OP}
\newtheorem{problem}{Problem}
\newtheorem{theorem}{Theorem}


\def\dt{{\rm d}t}
\def\dv{{\rm d}v}
\def\di{{\rm d}i}
\def\dI{{\rm d}I}
\def\dU{{\rm d}U}


\def\nat{{\hbox{\sf \tiny{nat}}}}
\def\nor{{\hbox{\sf \tiny{nor}}}}
\def\fb{\hbox{\tiny{\textsf FB}}}
\def\vione{{\hbox{\tiny{vi-1}}}}
\def\vitwo{{\hbox{\tiny{vi-2}}}}
\def\qvitwo{{\hbox{\tiny{qvi-2}}}}
\def\acone{{\hbox{\tiny{ac-1}}}}
\def\actwo{{\hbox{\tiny{ac-2}}}}





\begingroup
\count0=\time \divide\count0by60 % Hour
\count2=\count0 \multiply\count2by-60 \advance\count2by\time
% Min
\def\2#1{\ifnum#1<10 0\fi\the#1}
\xdef\isodayandtime{\the\year-\2\month-\2\day\space\2{\count0}:%
\2{\count2}}
\endgroup
%%%% fin macro %%%%


